\section{Weekly Recap}
    \subsection{Week 1}
        Week 1 was largely an introductory week, as required for the beginning of our first year-long project.
        This week was where we were introduced to the project options, and where we began determining what would best suit our interests, needs, and skill sets.
        We also got the opportunity to start getting in touch with clients to formally propose our interest in their projects and willingness to be on their team.
        As we started this presentation, this is where we began to formulate our reasons for joining the course, getting our degrees, and ultimately selecting our capstone project.
        We also submitted resumes and autobiographies to give the instructors and class a rough idea of our identities. 
    \subsection{Week 2}
        Week 2 was the initial selection of our projects, which was a fairly involved process for many in the course.
        A lot of thought and effort when into choosing which projects we would like request, and fortunately all of us were able to be on the Audio Extravaganza project.
        Our reasons for selecting this project revolved heavily around personal interest in music and/or audio engineering.
    \subsection{Week 3}
        Week 3 was where the ball started rolling for the actual formulation of our project and direction it would take.
        This started with the individual problem statements, in which we were asked to provide a high-level view of the “problem” our project was attempting to solve.
        Each of us took a different approach to defining the problem that was vaguely described by our client.
        While our problem can hardly be described as such, we were able to come to a conclusion after compiling each of our statements into one document, where we used the similar parts to define the following: we need new ways to make new sounds in a impressive format that is accessible, portable, affordable, and usable.
        One of the points of struggle this week was that the project description did not clearly define a problem.
        A significant amount of time was spent finding ways to spin the project description so that it could be described as a problem.
    \subsection{Week 4}
        Week 4 involved major decisions about the direction of the project, defining a more established workflow, and brainstorming.
        This resulted in a lot of confidence in our project and our group mates.
        We had our first client meeting where we got to gain a clearer understanding of the thoughts of Kirsten and Moog.
        We were able to successfully collaborate to write a cohesive problem statement that clearly communicated our big-picture ideas, even though we had not fully decided on the specifics of our plan.
        At this point, it was clear there was going to be some push and pull on which way this project was going to go, even on a high-level.
        But the group found compromise quickly and effectively that maintained everyone’s excitement and ideas for the project. 
    \subsection{Week 5}
        Week 5 was the first week where we saw a crunch for time, especially with midterms taking precedence for a large portion of our group.
        However, even with heavy assignment and other class loads, we sparked some great discussion about the general direction of the project.
        After our discussion, we felt much more confident in putting a dent in our requirements and tech documents without much hiccup.
    \subsection{Week 6}
        Week 6 involved focus on our requirements document and our tech reviews, which were both massive leaps into understanding the scope and ideas behind our project.
        While still a bit unclear on the specifics of our project, especially in the hardware realm, this was a point where information started to come together at a faster rate.
        Midterms have been taking priority for the previous two weeks, but we were still able to get some solid work in over the period in which we divided up our expertise areas for the tech review.
        At this point, we could start finalizing the tech review topics and documentation and meet to start thinking about implementation on a more detailed level.
        During this week, we also defined our team standards, which helped us to establish a more accountable and effective workflow. The standards have been helpful a great deal to our group. 
    \subsection{Week 7}
        Week 7 was a positive week that had us looking more into the specifics of what is needed on the technical side of our project.
        The tech reviews were massively insightful and allowed our team to develop roles more specifically that aligned with our research, experience, and desires.
        Next involved further diving into this area, and without a meeting on Monday due to the holiday we were prepared to be more individually minded, constructing our design document.
    \subsection{Week 8}
        Week 8 we talked about our tech reviews and reached out to Kevin about the purchase of hardware.
        This week was pivotal for us, because of our design documents due at the end of the break.
        This challenged us in a new way, as we needed to figure out an effective way to work together while being in many different areas for the holiday.
        As such, the week was not too productive with the break, but we found a way to remain effective. 
    \subsection{Week 9}
        Week 9 we submitted our design document, which was the largest commitment we made to the final approach to the implementation, we were all becoming anxious to begin the working on.
        Our ideas were beginning to manifest, with a clearer designation of roles and ideas for the final implementation. 
    \subsection{Week 10}
        Week 10 was largely a catch-up week, with strong conversations with our client and with Dr. McGrath about our technical decisions for the tech review.
        This was extremely valuable to us, although not necessarily encouraging as we found a fair amount of misinformation was produced in our tech reviews, mostly related to hardware, a scope of study that is not covered in our degree hardly at all.
        Luckily our resources are plenty and we’re confident in our ability to figure it out, and with meetings scheduled with Dr. McGrath, Sylvan from Moog, and our client, we are on track to be on the same page with our stakeholders and our group.
        From this point, it brings us up to the present day, where we have prepared our term progress report, and a corresponding presentation, sending us into a long break before hitting software implementation hard. 

\section{Retrospective}
Below, table \ref{tab:retro}, is a retrospective of our work over the past 10 weeks. Each row corresponds to a specific week, for example: information in row one correspond to week 1.
\begin{table}[!ht]
    \centering
    \begin{tabular}{|p{0.3\linewidth} | p{0.3\linewidth} | p{0.3\linewidth} |}
    \hline
    % wk 1 resume / start looking at projects
    % wk 2 selecting projects
    % wk 3 problem statement
    % wk 4 group problem statement
    % wk 5 tech review draft 1
    % wk 6 Requirements 1 // team standards
    % wk 7 Tech review final draft
    % wk 8 Design doc 1 workd
    % wk 9 Submitted design doc
    % wk 10 Meet with mcgrath
        \textbf{Positives}& \textbf{Deltas} & \textbf{Actions}  \\ \hline
        
        Had a resume workshop that gave insight into the skills we have developed during our time at Oregon State University.& None & None \\ \hline
        
        Sent in applications for projects and reached out to clients for information regarding their project.& None & None \\ \hline
        
        Met with our group, and finished our individual project statements.& There were conflicting visions about what our final product would be. & Organized a meeting to talk about our differing interpretations, and try to reach a mutual understanding for all members. \\ \hline
        
        Met with our client, shared our interpretations of the project, organized a line of communication via Slack, and wrote our final version of the problem statement.& We were not able to meet in person all that often once we started working on the paper. & Delegated duties via Slack so everybody knew what they were doing. \\ \hline
        
        Decided on topics for the Tech Review, and split them between all the group members.& Some of us were in the process of studying or taking midterms. & Did their work after their midterms were over. \\ \hline
        
        Wrote our Tech Review drafts and got them submitted. & Some of us were in the process of studying or taking midterms. & Did their work after their midterms were over. \\ \hline
        
        Got feedback from our client and peers about what should be included in the final version of the tech review, and applied the notes to our final versions.& None. & None. \\ \hline
        
        Started working on our design document.& Since it was the holidays, we did not meet with either our client or our TA & Used emails to communicate with them as necessary. \\ \hline
        
        Made significant progress on the design document.& With more holidays, it was difficult to work on the design document. & Once the break was over, we would put a large effort into finishing off the design document. \\ \hline
        
        Submitted our final version of the design document and began work on the end of term progress check-in. Talked with McGrath regarding hardware options and got feedback for the Tech Reviews.& A lot of the remaining content was finished the night is was due. & Decided to be more proactive when it came to writing projects, starting sooner. \\ \hline
        
    \end{tabular}
    \newline
    \caption{Fall 2018 Retrospective}
    \label{tab:retro}
\end{table}