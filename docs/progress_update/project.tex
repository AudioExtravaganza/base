\section{Project}

\subsection{Purpose}
The Audio Extravaganza project is centered around the creation of an impressive, manipulable, and intuitive tool to aid in the performance arts by creating an iconic effect in real time.
There is often a divide drawn between the artists, producers, and creators of a musical piece and the consumers, onlookers, and fans that surround the music. There are barriers between stages, different accounts for artists and users on distribution sites, and a completely hidden backend of the music industry that is often an afterthought for the common listener.
However, there is a large degree of overlap between the two groups, as creators are almost always avid listeners, and prohibitive factors such as financial means and technical expertise prevent people from creating music are receding as time goes on and technology progresses.
There is a lot of opportunity in designing projects that appeal to all groups across the spectrum of creators and consumers.
Products that are enjoyable, usable, versatile, and manipulable frequently yield results that are innovative and marketable; however, many products used today generally suffer from any combination of the following problems: high price point, complex interface, and portability.
We need new ways to make new sounds in a impressive format that is accessible, portable, affordable, and usable.

\subsection{Goals}
    The primary aim of this project is to build the \textit{Dam Good Pedal}, a modular digital effects processing pedal that receives and modifies the sound of an instrument or microphone paired with a looping module with which users can record and playback their input, allowing them to build complex rhythms and melodies with a single input. Once the base pedal unit is complete to our standards, we will develop an external interface to improve the learn-ability and usability of our platform.
    
\subsection{Current progress}
    The current state of the project is shifting between the planning and design phase to the implementation phase.
    So far, we have completed a problem statement, requirements document, individual technology reviews, and our design document.
    The problem statement clearly defines a problem, and our proposed solution. The requirements document clarifies client needs and defines them in a technical scope.
    The technology reviews examine unique technologies and compares available options to find what tools will be the most useful.
    Finally the design document defines our plans for implementation. All in all, we seem to be well prepared to begin implementation.

\subsection{Problems and challenges}
    This project has had a few problems and challenges.
    The first challenge we faced was adapting the project description into a meaningful problem statement.
    Each of us handled this differently, but we all managed to piece together a cohesive and meaningful problem statement.
    Another challenge we are facing is that our final product will rely heavily on hardware.
    This has made drafting up documentation that is originally designed to explain software systems a bit difficult to complete.
    The final challenge we have faced is working under time constraints, as many of the assignments have had fast turnarounds. This has left us working hard to get quality work out in short amounts of time. Overall, this experience is great preparation for working in industry where turnarounds can be quick, client demands can be vague, and systems can be foreign.