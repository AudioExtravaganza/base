\section{Introduction}
% Martin
\subsection{Purpose}
For our effects pedal, we will be developing a system to meet the needs of musicians and musical artists. Our product will need to operate in live musical performance environments, as well as in recording situations. Due to these requirements, it will be essential that out effects pedal can be operated easily and consistently with no technical hiccups, crashes, or unexpected results. Mitigating latency will also be crucial for developing a pedal capable of holding up in a recording environment, as well as ensuring that our system produces high quality audio with no hiss or buzz. 
Our stakeholders will primarily be people with a connection to music. Although music will be a shared trait, our users are expected to vary in terms of technical capabilities and experience using audio effects pedals. A common expectation for these users will be that our pedal is easy to operate, regardless of how experienced somebody is with effects pedals.\\
Our product's requirements will have to fit our stakeholders needs, in that we are able to create an easy-to-use yet powerful effects pedal capable of producing reliable audio that a performer can actually use during a concert. 

% Martin 
\subsection{Scope}
    \begin{enumerate}[label=\alph*.]
        \item Software product(s) to be produced by name: pedalProject - Final name to be determined at a later date.
        \item What it does: This effects pedal will improve the lives of musicians by providing an easy to use and reliable effects pedal which can hold up during live performances.
        \item Benefits/objectives/goals: Our pedal will provide benefits to the musician by giving them the ability to recursively apply effects to a looping sample, which can be changed to only affect certain previous or future loops. Our objectives are to extensively plan out our design, implement it in software, and to then move it to hardware. Our goals are to create the software for our system so that it is easily modifiable by the user, and allows for future modifications/additions into our pedal.
\end{enumerate}

\subsection{Product Overview}
    \subsubsection{Product Perspective}
    Our product will take influence from other currently available looping pedals, and will build on their designs to create something more user friendly and easily modifiable. 
    Seeing as effects pedals are often a part of larger effects chains, we will need to include audio input and output ports on our looping pedal, which our software will be able to interpret and output. 
    
    \subsubsection{Product Functions}
    Our product functions are as follows:
    \begin{enumerate}[label=\alph*.]
        \item Looping effect allowing the user to apply other effects to looping samples.
        \item The modulation of incoming audio through the application of audio effects.
        \item The addition for more effects to be added by the user easily.
    \end{enumerate}
    
    \subsubsection{User Characteristics}
    Our system will have users who are musicians with experience using multiple effects pedals, musicians with no experience using effects pedals, and non-musicians with a primary focus on the two preceding groups. The user will also act as a maintainer of the system, with the ability to apply more effects to the system based on our documentation. 
    
    \subsubsection{Limitations}
    There will be some hardware limitations, such as how much information we are able to store on a small single-board computer, as well as what audio quality we are able to achieve with and without using an external sound card. Reliability will be a limitation in that we need to focus on achieving a certain level of reliability above everything else. If our pedal has amazing sounding effects but the looping feature has unreliable latency or cuts out, it will not meet our standards for this project. 
    
% \subsection{Definitions}


