\section{Visual Output Devices}
    % Comparison Metrics
        % Ease of use
        % Color
        % Flexibility
        % Pins Required
    Visual output devices refers to displays and/or indicators that can be used to give user feedback on the state of the system.
    While there is a wide variety of visual output devices, the three types that will be evaluated are: LEDs, LCDs, and TFT LCDs. 
    The metrics used to evaluated each of these include: versatility, usability, and cost.
    \subsection{LED}
    Light emitting diodes, more commonly known as LEDs, are small individual lights. In electronics, they are often used to indicate a binary state of a component, like on or off. 
    
    In terms of versatility, LEDs can be used in a variety of different ways; however, they are limited in the types of information they can convey without infringing on usability.
    For example, an LED is good to show the rate of a specific event via blinking at the corresponding speed.
    They would not be good for showing information like an exact number or word.
    To maintain a usable output approach, an LED should only be used to display a basic state.
    Finally, In terms of cost, LEDs are relatively cheap. You can often find large quantities for a low price. 
    
    \subsection{Basic LCD}
    A Basic Liquid Crystal Display (LCDs) is a display, often multi-line, that can output letters and numbers on a screen.
    In terms of versatility, LCDs can display a broad scope of information, because they can display numbers and letters.
    These generally enhance usability, because they can show important information in a concise way.
    Some good use cases for these are: when a user needs to navigate a file system, or verify some complex information about the device.
    Finally, the cost of a basic LCD is also fairly low, roughly \$20.
    
    \subsection{TFT LCD}
    Thin film transistor liquid crystal displays (TFT LCD) are output devices that can show color images.
    The versatility of a TFT LCD is extremely high, because images and text can be shown to the user.
    This allows for a more robust method of displaying information.
    Often times, a TFT LCD will improve the usability of a device; however, it can often be overkill for many kinds of output.
    Finally, the cost of a small TFT LCD screen is moderate, but can increase based off of resolution and other capabilities, like touch sensitivity.
    
    \subsection{Concluding Thoughts}
        Each of these display types have unique use cases, which rely heavily on the kind of data that should be displayed to the user; furthermore, no single option will always be the right choice.
        These will have to be selected on a case by case basis. See table \ref{tbl:VOD} for generalized information.
        \begin{table}[!ht]
        \begin{center} 
        \caption{Visual Output Devices}
        \label{tbl:VOD}
            \begin{tabular}{| c | c | c | c | c |}
                \hline
                & Versatility & Usability & Cost & Best For \\ \hline
                \textbf{LED} & Low & High & Lowest & Displaying States \\ \hline
                \textbf{LCD} & Medium & High & Low & Alphanumeric Messages \\ \hline
                \textbf{TFT LCD} & High & High & Moderate & Any Type \\ \hline
            \end{tabular}
            
        \end{center}
        \end{table}            
