\documentclass{article}
\usepackage[utf8]{inputenc}

\title{Audio Extravaganza Problem Statement}
\author{Martin Barker}
\date{October 2018, CS461 Fall Term}


\usepackage{natbib}
\usepackage{graphicx}

\begin{document}

\maketitle

\section{Project abstraction}
The purpose of this project is to create a functional looping / sound modulation pedal for live performances that lives up to Moog's legacy. An effects pedal has many different purposes for musicians, for use in instances such as recordings and live performances. The world of hardware musical devices has many different communities with nostalgia for analog instruments and vintage devices, for this effects pedal we will try to pay homage to the legacy of looping/modulation pedals, while also striving for improvement and innovation. Our project will begin with brainstorming with teammates on what our final vision is. The two different fields a device like this could be used in, production and live musicianship, have different needs which can be catered to. Usability of the device will be an important focus, as well as determining if and how much we want to work on implementing features for digital audio workstations to use in cooperation with our physical pedal. Throughout our development, the innovation will be influenced by looking at other, similar looping/modulation products available on the market. Cataloging the different prices / features / learnability / functionality / design of looping and modulation pedals will help us to understand what the current market for these devices is like. After consulting this information, we will build our expectations for what our finished pedal should be able to do, as well as who the target demographic is. 




\section{Problem description}
Project description:
Musicians have alot of things to think about during a live performance, such as how they interact with their musical equipment. Most pedals have a simple on / off interface with various effects and filters, but looping pedals require a bit more attention due to their more precise nature. Looping pedals with muddled and confusing interfaces don't help with the performers attention to detail either, as there are always more important things to focus on rather then figuring out a pedal's confusing interface. Improving the lives of a musician will be the main goal of our project, especially focusing on the design, accessibility, learnability, and compatibility for our looping / modulation pedal. Improving how a musician interacts with our product will be the main objective, and will require some forethought into what challenges artists who use effects pedals face. Not only will this be achieved by catering to musician's concerns and feedback, but by also searching for new inventive sounds to push the creativity of a live performance. 

\section{Proposed solution}
Our proposed solution to this problem is to create a looping / modulation pedal with more emphasis on usability for the musician. The pedal will be built as something that is essential to the musicians live setup, and easily interfaces with their audio input and output. With simplicity as a primary focus, the pedal will become something that a musician prefers to use as opposed to other looping pedals available on the market. Addressing an artists needs for both live performances and recording will be taken into account for how the pedal address issues of complexity, making sure that it remains functional and sonically innovative. When an artists reaches for the pedal to change something mid performance, their experience using the pedal is what will determine how effective it is. Measuring things such as the time they spend to achieve a certain outcome will help us to decide which features need to be improved. The ultimate goal will be to improve the ease of access a user has when using the pedal, and to make sure it takes the least amount of time possible to achieve the desired sound. Efficiency is something which many other pedals lack, but is incredibly important for situations in which timing is important (such as live performances). A pedal with poor efficiency can lead to live performances in which the musician is not able to achieve a desired sound fast enough, which can impair the performance. This is something that we hope to avoid with our pedal.

\section{Performance metrics}
In order to measure our performance on this project, we will be looking at many different elements during the design and implementation process. Ultimately our end product will need to be used by a musician, so examining metrics such as efficiency and usability will be the most important. In order to accurately measure the success of our effects pedal, we will need to enlist the help of musicians during the hardware stages of our project. These musicians can help field test our pedal and give us feedback as to the sound quality of the product, as well as how easy it is to use. Overall performance of our group can be measured at various steps throughout the next year, mainly divided up into three sections; brainstorming / design, programming, and hardware. Success throughout these three steps will be measured in different ways. The first step, brainstorming / design, can be deemed successful once our team decides on a design to pursue, and we are each on the same page about what we would like to accomplish by the end of the year. We should come to a consensus on what exactly we expect our finished product to be able to do, and design a road map of the various problems we need to tackle in the coming terms. The next step, programming, will involve programming our pedal to complete various audio effects. This stage will be mostly carried out on the computer, but can be measured as successful when we can successfully create a digital version of our pedal that manipulates the sound in a desired way. After this point, we need to handle creating a physical device capable of carrying our software. Success in this step can be measured by examining our prototypes, and how well they are able to hold up during a live performance. Early prototypes should only focus on successfully producing the desired audio outcome, while as our prototypes improve we should spend more time on usability and creating a learnable product. 
\end{document}
