\section{Solution, Tools, and Methods}
    %   Digitized modular synthesizer
    %   Each module can extend behaviors
    \subsection{Solution}
        % The solution I propose is an open-source digitally implemented dynamic modular system.
        % This device would be modular in the sense that one module can be standalone, or paired with multiple modules to create a more diverse system.
        % Each module would be dynamic, meaning that it could shift from one state to another in separate patches. For example, in one patch it could work as an LFO\footnote{Low frequency oscillator.}, but in the other it could work as a filter.
        % Because this would be an open-source implementation, it would allow a community of users to manipulate our framework to design and share new modules.
    
        % This solution solves many of the problems listed above. Firstly, it helps reduce the total cost of a modular system, because one module can be used to achieve many different things. %better word than things
        % Secondly, it will be portable, because one module can replicate many modules.
        
        The solution I propose is an open-source semi-modular synthesizer embedded on a small device.
        This device's main goal would be to bring some of the creative inspiration of a modular synthesizer to a small, portable device.
        From a high level, the device would simulate a variety of modules while giving users a physical interface to tweak their behavior.
        This device would also be open-source, so users could create and use their own modules in patches.
        
        Some key features of this device would be:
        \begin{itemize}
            \item A User will be able to interact with the synthesizers modules using hardware inputs like dials, sliders, buttons, and switches.
            \item Users will be able to create their own modules and patches on a computer and import them onto the device via a standard data transfer method like access to the file system via USB.
            \item Users will be able to save and reload the state of the device to re-use a patch later. This will significantly help during performances where the artist has an extremely complex patch.
            \item Additional external input will be optional. I.e The device will generate audio on its own without the need for an outside signal.
            \item Users will be able to use midi controllers as additional external input, to allow musicians to use devices like a keyboard or launchpad.
            \item Users will be able to connect output devices via standard audio output.
            \item The device would be usable and engaging for beginners and experts alike.
        \end{itemize}
    
    \subsection{Tools \& Methods}
        To implement this solution, the team will make use of a variety of audio synthesis frameworks and hardware tools.
		One of the main frameworks we will use is SuperCollider, which is an open-source "platform for audio synthesis and algorithmic compositions"\cite{sc}.
		Another framework we will consider using is PureData, which "is an open source visual programming language for multimedia"\cite{pd}.
		These two tools will be used to create a working digital prototype on systems that have standard hardware specifications.
        Once we have a working digital prototype, we will test its performance to verify that it will run smoothly on a system with significantly lower hardware specifications.
        Once our tests are successful, we will embed our solution on a small board.
        There are a few constraints we will have to consider when choosing the board to use.
        First, it must have sufficient processing power.
        It will also need to have the availability for a large set of general purpose input/output (GPIO) pins.
        Therefore, some of the small devices we will likely explore the Beagleboard Black and Raspberry Pi 3.
        Finally, we will add hardware inputs to allow the user to interact with the software in an intuitive way that does not inhibit the creative process.
