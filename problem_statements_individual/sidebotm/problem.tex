\section{Problem Statement}
    % 	I'm working on a musical collaboration. My goal is to wow the musical artists I work with and also produce a crowd-pleasing expo experience for visitors to OSU. I have a small idea of what's possible, but I expect that the right team would innovate beyond the initial requirements/vision to create something awesome and functional for live performances.
    
    %   Who     - Musicians.
    %   What    - Large devices like a modular synthesizer rack are extremely large to travel with.
    %               Advanced tools are expensive, large, and hard to learn.
    
    %   When    - Live Performances.
    %   Where   - Performance venues.
    %   why     - If this problem were fixed, musicians would have to opportunity to use inspiring tools and instruments in their performances.
    
    % Team shall collect requirements, research potential innovation to initial vision, develop product on laptop/desktop or Beaglebone singleboard computer, migrate product to dedicated portable hardware, test with members of musical collaboration--both novices and experts, add crowd-pleasing 'sparkle' (or funk--whichever the team prefers) at every step
    
    %benefits they are extremely complex and have many outcomes
    
    In live performances, musicians rarely use one modular synthesizers, which are one of the most intricate and awe-inspiring instruments.
    An example of one of the greatest implementations of this instrument is seen in musician Deadmau5's studio\cite{mau5trap}, where he has hundreds of modules built into various racks. 
    These modular synthesizers are used to create an extremely diverse, and nearly infinite set of sounds that are difficult to create using other methods.
    In fact, when demonstrating his modular synthesizer system, Joel Zimmerman stated, "the nature of these things is they're really hard to tune and you're always going to get minor tuning variances... but that's going to give you a kind of sound that you cant replicate with a VST\footnote{Virtual Studio Technology.
    I.e a digital implementation used in audio production.}"\cite{mc:deadmau5}
    
    Unfortunately, modular synthesizers are challenging to use in performances for a variety of reasons.
    To start, they are expensive instruments, which makes it tough for an artist to risk damaging one in transit or during a live performance.
    Secondly, due to the fact that there are limitless patches\footnote{Specific configurations on modular synthesizers.} they are sometimes overwhelming and complex to learn.
    Finally, they are not generally portable, because they frequently are built into large racks, which containing large amounts of components.
    
    A variety of current solutions exist, but they leave a lot to be desired.
    For example, there are many applications for mobile devices that simulate the behavior of a modular synthesizer system.
    Being on mobile devices adds many limitations to the flexibility and diversity of a modular synthesizer. Furthermore, they can be less inspiring for musicians.
    Other systems also exist, like a semi-modular synthesizer; however, they tend to still be expensive.
    Other solutions include sampling a patch to re-use later.
    
    
    
