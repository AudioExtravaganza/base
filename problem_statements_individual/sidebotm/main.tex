\documentclass[onecolumn, draftclsnofoot,10pt, compsoc]{IEEEtran}
\usepackage{graphicx}
\usepackage{url}
\usepackage{setspace}
\usepackage{cite}
\usepackage{geometry}
\geometry{textheight=9.5in, textwidth=7in}

% 1. Fill in these details
\def \CapstoneTeamName{		Audio Extravaganza Team}
\def \CapstoneTeamNumber{		26}
\def \GroupMemberOne{			Mason Sidebottom}
\def \CapstoneProjectName{		Audio Extravaganza}
\def \CapstoneSponsorCompany{	Audio Extravagnaza Client}
\def \CapstoneSponsorPerson{		Dr. Kirsten Winters}

% 2. Uncomment the appropriate line below so that the document type works
\def \DocType{		Problem Statement
				%Requirements Document
				%Technology Review
				%Design Document
				%Progress Report
				}
			
\newcommand{\NameSigPair}[1]{\par
\makebox[2.75in][r]{#1} \hfil 	\makebox[3.25in]{\makebox[2.25in]{\hrulefill} \hfill		\makebox[.75in]{\hrulefill}}
\par\vspace{-12pt} \textit{\tiny\noindent
\makebox[2.75in]{} \hfil		\makebox[3.25in]{\makebox[2.25in][r]{Signature} \hfill	\makebox[.75in][r]{Date}}}}
% 3. If the document is not to be signed, uncomment the RENEWcommand below
\renewcommand{\NameSigPair}[1]{#1}

%%%%%%%%%%%%%%%%%%%%%%%%%%%%%%%%%%%%%%%
\begin{document}
\begin{titlepage}
    \pagenumbering{gobble}
    \begin{singlespace}
        \hfill  
        \par\vspace{.2in}
        \centering
        \scshape{
            \huge CS Capstone \DocType \par
            {\large\today}\par
            \vspace{.5in}
            \textbf{\Huge\CapstoneProjectName}\par
            \vfill
            {\large Prepared for}\par
            \Huge \CapstoneSponsorCompany\par
            \vspace{5pt}
            {\Large\NameSigPair{\CapstoneSponsorPerson}\par}
            {\large Prepared by }\par
            Group\CapstoneTeamNumber\par
            % 5. comment out the line below this one if you do not wish to name your team
            % \CapstoneTeamName\par 
            \vspace{5pt}
            {\Large
                \NameSigPair{\GroupMemberOne}\par
            }
            \vspace{20pt}
        }
        \begin{abstract}
        % 6. Fill in your abstract    
		Many musicians, like Joel Zimmerman, that use modular synthesizers in recorded pieces do not use them during live performances.
		For the most part, this is due to logistical challenges associated with the devices.
		Currently, a small set of solutions exist; however, they leave a lot to be desired, because they lack many of the key elements that make modular synthesizers so unique.
		To help fix this problem, I propose creating an open-source semi-modular synthesizer that attempts to bridge the gap between modular synthesizers and the current attempts at solutions.
		To implement this, we will use a variety of open-source software frameworks and general purpose hardware.
		This solution aims to lower costs, be usable for beginners and experts, and work smoothly during performances.
		
		\end{abstract}     
    \end{singlespace}
\end{titlepage}
\newpage
\pagenumbering{arabic}
\tableofcontents
% 7. uncomment this (if applicable). Consider adding a page break.
%\listoffigures
%\listoftables

% Uncomment to make table of contents full page
\clearpage

% 8. now you write!
\section{Problem Statement}
    % 	I'm working on a musical collaboration. My goal is to wow the musical artists I work with and also produce a crowd-pleasing expo experience for visitors to OSU. I have a small idea of what's possible, but I expect that the right team would innovate beyond the initial requirements/vision to create something awesome and functional for live performances.
    
    %   Who     - Musicians.
    %   What    - Large devices like a modular synthesizer rack are extremely large to travel with.
    %               Advanced tools are expensive, large, and hard to learn.
    
    %   When    - Live Performances.
    %   Where   - Performance venues.
    %   why     - If this problem were fixed, musicians would have to opportunity to use inspiring tools and instruments in their performances.
    
    % Team shall collect requirements, research potential innovation to initial vision, develop product on laptop/desktop or Beaglebone singleboard computer, migrate product to dedicated portable hardware, test with members of musical collaboration--both novices and experts, add crowd-pleasing 'sparkle' (or funk--whichever the team prefers) at every step
    
    %benefits they are extremely complex and have many outcomes
    
    In live performances, musicians rarely use one modular synthesizers, which are one of the most intricate and awe-inspiring instruments.
    An example of one of the greatest implementations of this instrument is seen in musician Deadmau5's studio\cite{mau5trap}, where he has hundreds of modules built into various racks. 
    These modular synthesizers are used to create an extremely diverse, and nearly infinite set of sounds that are difficult to create using other methods.
    In fact, when demonstrating his modular synthesizer system, Joel Zimmerman stated, "the nature of these things is they're really hard to tune and you're always going to get minor tuning variances... but that's going to give you a kind of sound that you cant replicate with a VST\footnote{Virtual Studio Technology.
    I.e a digital implementation used in audio production.}"\cite{mc:deadmau5}
    
    Unfortunately, modular synthesizers are challenging to use in performances for a variety of reasons.
    To start, they are expensive instruments, which makes it tough for an artist to risk damaging one in transit or during a live performance.
    Secondly, due to the fact that there are limitless patches\footnote{Specific configurations on modular synthesizers.} they are sometimes overwhelming and complex to learn.
    Finally, they are not generally portable, because they frequently are built into large racks, which containing large amounts of components.
    
    A variety of current solutions exist, but they leave a lot to be desired.
    For example, there are many applications for mobile devices that simulate the behavior of a modular synthesizer system.
    Being on mobile devices adds many limitations to the flexibility and diversity of a modular synthesizer. Furthermore, they can be less inspiring for musicians.
    Other systems also exist, like a semi-modular synthesizer; however, they tend to still be expensive.
    Other solutions include sampling a patch to re-use later.
    
    
    

\section{Solution, Tools, and Methods}
    %   Digitized modular synthesizer
    %   Each module can extend behaviors
    \subsection{Solution}
        % The solution I propose is an open-source digitally implemented dynamic modular system.
        % This device would be modular in the sense that one module can be standalone, or paired with multiple modules to create a more diverse system.
        % Each module would be dynamic, meaning that it could shift from one state to another in separate patches. For example, in one patch it could work as an LFO\footnote{Low frequency oscillator.}, but in the other it could work as a filter.
        % Because this would be an open-source implementation, it would allow a community of users to manipulate our framework to design and share new modules.
    
        % This solution solves many of the problems listed above. Firstly, it helps reduce the total cost of a modular system, because one module can be used to achieve many different things. %better word than things
        % Secondly, it will be portable, because one module can replicate many modules.
        
        The solution I propose is an open-source semi-modular synthesizer embedded on a small device.
        This device's main goal would be to bring some of the creative inspiration of a modular synthesizer to a small, portable device.
        From a high level, the device would simulate a variety of modules while giving users a physical interface to tweak their behavior.
        This device would also be open-source, so users could create and use their own modules in patches.
        
        Some key features of this device would be:
        \begin{itemize}
            \item A User will be able to interact with the synthesizers modules using hardware inputs like dials, sliders, buttons, and switches.
            \item Users will be able to create their own modules and patches on a computer and import them onto the device via a standard data transfer method like access to the file system via USB.
            \item Users will be able to save and reload the state of the device to re-use a patch later. This will significantly help during performances where the artist has an extremely complex patch.
            \item Additional external input will be optional. I.e The device will generate audio on its own without the need for an outside signal.
            \item Users will be able to use midi controllers as additional external input, to allow musicians to use devices like a keyboard or launchpad.
            \item Users will be able to connect output devices via standard audio output.
            \item The device would be usable and engaging for beginners and experts alike.
        \end{itemize}
    
    \subsection{Tools \& Methods}
        To implement this solution, the team will make use of a variety of audio synthesis frameworks and hardware tools.
		One of the main frameworks we will use is SuperCollider, which is an open-source "platform for audio synthesis and algorithmic compositions"\cite{sc}.
		Another framework we will consider using is PureData, which "is an open source visual programming language for multimedia"\cite{pd}.
		These two tools will be used to create a working digital prototype on systems that have standard hardware specifications.
        Once we have a working digital prototype, we will test its performance to verify that it will run smoothly on a system with significantly lower hardware specifications.
        Once our tests are successful, we will embed our solution on a small board.
        There are a few constraints we will have to consider when choosing the board to use.
        First, it must have sufficient processing power.
        It will also need to have the availability for a large set of general purpose input/output (GPIO) pins.
        Therefore, some of the small devices we will likely explore the Beagleboard Black and Raspberry Pi 3.
        Finally, we will add hardware inputs to allow the user to interact with the software in an intuitive way that does not inhibit the creative process.

\section{Performance Metrics}
The main goal of this project is to help bring tools like modular synthesizers into live performances; therefore, our performance metrics will revolve around the impact it has on musicians.
Currently, these are the criteria that are the most important:
\begin{itemize}
    \item \textbf{Cost:}
        The device must cost at most 200 USD to create. If the price exceeds this limit, the device would no longer be viable as a cheaper alternative.
        
    \item \textbf{Performance:}
        The device must be able to run a large amount of modules without slowing down; otherwise, the timing of a musicians piece would be compromised.
        
    \item \textbf{Usability:}
        The device must be user friendly to the point that an inexperienced user is able to successfully create interesting sounds. It must also provide more experienced users with enough functionality to keep them engaged.
\end{itemize}




\clearpage

\bibliographystyle{IEEEtran}
\bibliography{./refs.bib}
\end{document}
