%\section{Appendices}

%\subsection{Assumptions and Dependencies}
    %add \label{some name}
    % use \ref{labelname} in text to refer to here
\subsection{Acronyms and Abbreviations}
    The following abbreviations are used within our document:
    \begin{itemize}
        \item \label{DAC} \textbf{DAC:} Digital Analog Converter.
        %\item \label{FPGA} \textbf{FPGA: } Field-Programmable Gate Array
        \item \label{GPIO} \textbf{GPIO:} General Purpose Input/Output
        \item \label{LCD} \textbf{LCD:} Liquid-Crystal Display.
        \item \label{LED} \textbf{LED:} Light-Emitting Diode.
        \item \label{MIDI} \textbf{MIDI:} Musical Instrument Digital Interface.
        \item \label{MSRP} \textbf{MSRP:} Manufacturer's Suggested Retail Price.
        \item \label{USB} \textbf{USB:} Universal Serial Bus
        \item \label{UML} \textbf{UML:} Unified Modeling Language
    \end{itemize}

\subsection{Key Terminology}
    The following terms are used to describe tools, and design choices that need additional explanation:
    \begin{itemize}
        \item Digital Analog Converter: Converts a digital signal (binary ones and zeros) into an analog signal (electrical high voltage and low voltage) for use when outputting audio from our device.
        \item Patches: The different audio effects that our product uses. As an example, our looper and our modulation effects are two separate patches that can be used either individually or together.
        \item UML Class Diagram: Diagram that shows the relationship between different classes and data types. Is used to describe the structure and properties of a system.
        
    \end{itemize}