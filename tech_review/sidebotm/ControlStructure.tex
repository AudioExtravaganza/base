\section{Process Control Languages}
Process control languages refers to the languages that can be used to create underlying software between the operating system and our modulation software.
This will manage, verify, and control the file system, dependencies, and tasks.
This component will be crucial to making sure that all processes are started automatically after powering up the pedal.
There are many different languages that can be used to achieve this; however, the languages that will be examined are bash scripts, python3, and C++. 
The metrics used to compare them will be speed, syntax, and support.
Speed will be evaluated with large, complex programs in mind.
Syntax will be measured with maintenance and understandability in mind.
Finally, support is a measure of the access to helpful community threads and documentation.
    %Comparison Metrics
        % Speed
        %
        % Re write this section
        
        % Maintainability
        
        
    \subsection{Bash Scripting}
        The Bourne Again Shell, more commonly known as Bash, is a shell that conforms to the "IEEE POSIX P1003.2/ISO 9945.2 Shell and Tools standard" licensed under the GNU Public License \cite{bib:bash}.
        It was designed to group "features from the Korn Shell and C Shell" into one shell.
        It highlighted and improved features like: shell functions, job control, aliases, and more\cite{bib:bash}.
        Bash scripts allow you to create a program that calls other programs sequentially as well introduce useful utilities.
        
        In terms of speed, bash scripts are quick for operating system commands like \emph{ls} and \emph{dir}.
        However, there are a few factors that negatively impact the speed of bash scripts, like the fact that bash is interpreted\cite{bib:bashMan}.
        Most importantly, the efficiency of a bash script relies heavily on the programmer's experience and knowledge of the language.
        This is mostly due to the fact that there are many common practices that spawn new sub-processes and add overhead.
        Overall, bash performs fine by this metric, especially for basic programs; however, it begins to fall short when the complexity of a program grows.
        
        Bash script syntax can be extremely challenging to understand, because of the idea that there are many different ways to accomplish one task.
        The significance of this problem generally increases as the size and complexity of a program grows.
        Finally, it is incredibly challenging to collaborate on a bash script for many of the reasons listed above.
        
        The amount of support for Bash scripting is relatively high.
        Unfortunately, many community threads introduce new syntax, which can create more of a problem.
        Fortunately, the full manual is accessible online.

    \subsection{Python3}
        
        Python3 is an open source, interpreted programming language that has applications across the spectrum, ranging from application development to scientific analysis \cite{bib:py}. 
        Python can use a package manager named pip, which allows users to install other modules and libraries\cite{bib:pip}.
        
        In terms of performance, python3 is a quick language that does a lot of optimization behind the scenes.
        Even though it is interpreted, many of the libraries that are used will be byte compiled, further increasing its performance.
        
        One of the strongest suits of python3 is its level of maintainability.
        It boasts simple, but elegant syntax, that makes it very easy to read a python script and follow along.
        The design principles discussed in \textit{The Zen of Python} explain that code written in python should always strive to be understandable, simple, and explicit\cite{bib:pep20}.
        Therefore, python performs extremely well in terms of syntax.
        
        Finally, the support for python3 is extremely good.
        The documentation online is very detailed and understandable.
        Online community threads generally contain helpful information that is easy to understand and adapt.
    \subsection{C++}
        C++ is a language created by Bjarne Stroustrup in 1979\cite{bib:c++H}. It is a compiled language that gives immense control to the programmer at the cost of being unsafe \cite{bib:c++D}.
        In terms of speed C++ is widely regarded as one of the fastest languages available, because it is compiled to machine code\cite{bib:c++D}.
        For the most part, C++ syntax is relatively understandable;
        However, for programmers who haven't used C++ before, there can be a significant learning curve with memory management and identifying and fixing segmentation faults.
        In a few cases, code can become extremely hard to understand, especially if that is the developers goal. 
        That being said, this is not a problem that is unique to just C++, and can be seen in nearly all languages.
        The support for C++ is very verbose, and it is very easy to find helpful information on either the online documentation or community threads. 
    \subsection{Concluding Thoughts}
        While these three languages can be helpful for specific tasks, python3 and C++ are both great candidates.
        If the hardware performance is low, C++ should be used to maximize efficiency.
        On the other hand, if hardware performance is fairly good, python3 is a good selection. See table \ref{tbl:PCL} for generalized information.
        \begin{table}[!ht]
            \begin{center}
            \caption{Process Control Languages}
            \label{tbl:PCL}
                \begin{tabular}{| c | c | c | c | c | c |}
                    \hline
                    & Speed & Interpreted & Compiled & Syntax & Support \\ \hline
                    \textbf{Bash Scripts} & Medium & Yes & - & Hard & Moderate \\ \hline
                    \textbf{Python3} & Medium to Fast & Yes & Can Byte Compile & Easy & Good \\ \hline
                    \textbf{C++} & Fastest & No & Machine Code & Moderate & Good \\ \hline
                \end{tabular}
            \end{center}
        \end{table}
            